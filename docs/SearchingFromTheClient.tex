\documentclass[11pt,letterpaper]{article}
\usepackage[utf8]{inputenc}
\usepackage[margin=1.0in]{geometry}
\usepackage{float}
\usepackage{hyperref}
\usepackage{graphicx}
\usepackage{fancyvrb}
\setlength{\parskip}{\baselineskip}
\hypersetup{
    colorlinks=true,
    linkcolor=blue,
    filecolor=magenta,
    urlcolor=blue,
}

\begin{document}
\noindent From the client, you can access search functionality from the url 

\url{http://cam-dorm.tk/search/<latitude1>/<longitude1>/<latitude2>/longitude<2>} 

\noindent where the first set of coordinates refers to your current location and the second set refers to where you want to go. For example, 

\url{http://cam-dorm.tk:8000/search/-72.2487053/41.8040168/-72.2603938/41.8031481} 

\noindent gets directions from South Campus to Next Gen. The server returns the search results in JSON. That format is, as per Omar's documentation:

\begin{verbatim}
{(([start,distance from startloc][end,distancefrom endloc]):{route:time}}
\end{verbatim}

\noindent The directions I got from South to Next Gen were:
\begin{verbatim}
{(('Arjona Westbound', 0.0014533417801745336),
('Hilltop Community Center' 0.00048495324517030413)): {'Blue': 2023},
(('Arjona Eastbound', 0.001429215704507714), 
('Hilltop Dorms Southbound',0.0025955969737216124)): {'Weekend': inf}, 
(('Arjona Westbound', 0.0014533417801745336), 
('Hilltop Dorms Southbound',0.0025955969737216124)): {}, 
(('Arjona Eastbound', 0.001429215704507714), 
('Hilltop Community Center', 0.00048495324517030413)): {'Blue': 2023, 'Weekend': inf}}
\end{verbatim}
\end{document}